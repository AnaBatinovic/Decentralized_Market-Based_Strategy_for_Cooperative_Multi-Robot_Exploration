%%%%%%%%%%%%%%%%%%%%%%%%%%%%%%%%%%%%%%%%%%%%%%%%%%%%%%%%%%%%%%%%%%%%%%%%%%%%%%%%
%2345678901234567890123456789012345678901234567890123456789012345678901234567890
%        1         2         3         4         5         6         7         8

\documentclass[letterpaper, 10 pt, conference]{ieeeconf}  % Comment this line out
                                                          % if you need a4paper
%\documentclass[a4paper, 10pt, conference]{ieeeconf}      % Use this line for a4
                                                          % paper

\IEEEoverridecommandlockouts                              % This command is only
                                                          % needed if you want to
                                                          % use the \thanks command
\overrideIEEEmargins
% See the \addtolength command later in the file to balance the column lengths
% on the last page of the document

\pdfminorversion=4

% The following packages can be found on http:\\www.ctan.org
%\usepackage{graphics} % for pdf, bitmapped graphics files
%\usepackage{epsfig} % for postscript graphics files
%\usepackage{mathptmx} % assumes new font selection scheme installed
%\usepackage{times} % assumes new font selection scheme installed
%\usepackage{amsmath} % assumes amsmath package installed
%\usepackage{amssymb}  % assumes amsmath package installed
\usepackage{amsmath}    			% ams packages for mathematics environment    
\usepackage{amssymb}
\usepackage{amsfonts}
%\usepackage{amsthm}

\usepackage{graphicx}  				% Versatile graphics manipulation options

\usepackage[croatian]{babel}  % Croatian typographical rules and hyphenation patterns 
\usepackage[utf8]{inputenc}  	% Encoding of Croatian characters
\usepackage[T1]{fontenc}
\usepackage{ae,aecompl}     	% Type 1 fonts, similar to Computer Modern

\usepackage{microtype}				% Improves spacing

\usepackage{subfig}
\usepackage{subcaption}

\usepackage{tabularx}
\usepackage{booktabs}
%\newcolumntype{C}{>{\centering\arraybackslash}X} % centered version of "X" type
\setlength{\extrarowheight}{1pt}
\usepackage{enumerate}				% Additional options for listing of items in enumerate environment
\usepackage{todonotes}				% Adding todo items
\usepackage{dirtree}					% Simple display of directory tree
\usepackage{hyperref}					% Managing cross-referencing


\usepackage{lmodern}
\usepackage{nccmath}
\usepackage[keeplastbox]{flushend}
\usepackage{scalerel,stackengine}
\stackMath
\newcommand\reallywidehat[1]{%
	\savestack{\tmpbox}{\stretchto{%
			\scaleto{%
				\scalerel*[\widthof{\ensuremath{#1}}]{\kern.1pt\mathchar"0362\kern.1pt}%
				{\rule{0ex}{\textheight}}%WIDTH-LIMITED CIRCUMFLEX
			}{\textheight}% 
		}{2.4ex}}%
	\stackon[-6.9pt]{#1}{\tmpbox}%
}
\parskip 1ex

%Additional packages
\usepackage{mathptmx}
\usepackage{graphicx, subcaption}
\usepackage{array}
\usepackage{booktabs}
\usepackage[ruled, vlined, linesnumbered]{algorithm2e}
\usepackage{cleveref}
\renewcommand{\algorithmautorefname}{Algorithm}
\SetKwInput{KwPersistent}{Persistent variables}
\SetKwInput{KwLocal}{Local variable initializations}
\SetArgSty{textnormal}
\makeatletter
\newcommand{\nosemic}{\renewcommand{\@endalgocfline}{\relax}}% Drop semi-colon ;
\newcommand{\dosemic}{\renewcommand{\@endalgocfline}{\algocf@endline}}% Reinstate semi-colon ;
\newcommand{\pushline}{\Indp}% Indent
\newcommand{\popline}{\Indm\dosemic}% Undent
%\let\oldnl\nl% Store \nl in \oldnl
%\newcommand{\nonl}{\renewcommand{\nl}{\let\nl\oldnl}}% Remove line number for one line
%\newcounter{algoline}
%\newcommand\Numberline{\refstepcounter{algoline}\nlset{\thealgoline}}
%\AtBeginEnvironment{algorithm}{\setcounter{algoline}{0}}
\makeatother

\makeatletter
\newcommand{\removelatexerror}{\let\@latex@error\@gobble}
\makeatother


%calligraphy packages
\usepackage{calrsfs}
\DeclareMathAlphabet{\pazocal}{OMS}{zplm}{m}{n}
\newcommand{\Ca}{\pazocal{C}}
\newcommand{\Oa}{\pazocal{O}}
\newcommand{\Va}{\pazocal{V}}
\newcommand{\Ua}{\pazocal{U}}
\newcommand{\Aa}{\pazocal{A}}
\newcommand{\Ta}{\pazocal{T}}
\newcommand{\La}{\pazocal{L}}

\newcommand{\Ja}{\pazocal{J}}

\graphicspath{{./figures/}}
\usepackage{float}

\title{\LARGE \bf
Decentralized Strategy for Cooperative Multi-Robot Exploration and Mapping
}
\author{Ana Batinovi\'{c}, Juraj Or\v{s}uli\'{c}, Tamara Petrovi\'{c}, Stjepan Bogdan
	\thanks{Authors are with the University of Zagreb, Faculty of Electrical Engineering  and Computing, LARICS Laboratory for Robotics and Intelligent Control Systems, Unska 3, 10000 Zagreb, Croatia
        {\tt\small (ana.batinovic, juraj.orsulic, tamara.petrovic, stjepan.bogdan) at fer.hr}}}

\makeatletter
\newcommand{\removelatexerror}{\let\@latex@error\@gobble}
\newcommand{\mb}[1]{\boldsymbol{#1}}
\newcommand{\norm}[1]{\left\lVert#1\right\rVert}

\makeatother

\begin{document}
\maketitle

\thispagestyle{empty}
\pagestyle{empty}


%%%%%%%%%%%%%%%%%%%%%%%%%%%%%%%%%%%%%%%%%%%%%%%%%%%%%%%%%%%%%%%%%%%%%%%%%%%%%%%%
\begin{abstract}

This work presents a novel approach to autonomous decentralized multi-robot frontier exploration and mapping of an unknown area.  The mobile robot team simultaneously explores the environment, discovers frontier points (points on the border between the explored and unexplored space), and communicates by fully connected graph and event-based communication in order to become dispersed throughout the environment. During the exploration, the amount of information exchanged between the robots is limited to data about the robot positions and their current target points. The main goal is to allocate the mobile robots to target frontier points in the environment in a way which minimises the overall exploration time. Moreover, mobile robots creates common map of the environment. The proposed strategy has been implemented in a simulation environment and compared with a coordinated exploration strategy. The simulation results demonstrate the effectiveness of the proposed decentralized multi-robot strategy.

\end{abstract}
\section{INTRODUCTION}
Multi-robot systems belong to core robotics problems that have drawn significant attention in last few decades. The coordination of a mobile robot team during the exploration of an unknown area is a common problem encountered in many applications, such as search and rescue \cite{Murphy2004}, cleaning \cite{Endres}, \cite{Pinheiro2015}, warehousing \cite{Wurman2008}, or planetary exploration \cite{Mataric2001}, to name a few. Due to the fact that autonomous multi-robot systems are entering society domain and as such will interact with people on daily basis, development of efficient coordination algorithms becomes inevitable.

 Like in the human world, robots can be more effective when they work together. Moreover, a robot team can accomplish a predefined task much quicker than a single robot can \cite{Dias2000}. Another advantage of mobile robot teams arises from the possibility of merging overlapping sensor information, which in turn can help to compensate for sensor uncertainty \cite{Wurm2008}.
If is done properly, multi-robot coordination can lead to i) task accomplishment in the shortest possible time, ii) increased robustness, iii) higher map quality (in case of exploration task), and finally iv) the completion of tasks impossible for a single robot to perform \cite{Dias2006}.

\begin{figure}[t!]
    \centering
	\fbox{\includegraphics[width=0.85\columnwidth]{./Pictures/rviz_environment.pdf}}
	\caption {The description of the environment. A 2D map is represented with an occupancy grid that divides the map into cells: the white cells describe explored and grey cells unexplored space, while the black cells define obstacles. The frontier detector publishes frontier points (red) and the filter module publishes filtered frontier points (green).}
	\label{fig:environment}
\end{figure}


We consider the problem of autonomous multi-robot exploration and mapping using a fast dense frontier detector and a new decentralized exploration strategy. A result of frontier detection are points on the border of the explored and unexplored space in the environment - \textit{frontier points} shown in the Fig. \ref{fig:environment}. It is a dense set of frontier points, thus we use a filter to get a more manageable number of the frontier points. Since the main goal is to achieve faster exploration and better coordination among the mobile robots, our decentralized strategy ensures that the mobile robots become dispersed throughout the environment using a frontier occupancy function which takes into account current assigned points and position of all other mobile robots in exploration and mapping process. Every team member also calculates the cost and the utility of reaching all frontier points. Finally, the Hungarian algorithm \cite{Kuhn1955}, that attempts to find an optimal assignment solution (a target point for a mobile robot), is run by each mobile robot in parallel. It is assumed that communication among the mobile robots is modelled by a fully connected (complete) graph in order to avoid sub-optimal results as a consequence of information missing. 
If we consider connected communication graph but may not complete, then we need to think about a decentralized map generation, what is currently not our scope. An additional novelty of this work is the event based communication between mobile robots and minimum exchanging information about mobile robot positions and target points.  When a mobile robot is assigned to a frontier point for exploration, the algorithm for path planning and following navigates the mobile robot toward the target point. The process is over when the whole area is mapped.

The exploration and mapping strategy is implemented on multi-robot system using ROS framework. The strategy is compared to coordinated Burgard's Algorithm 1 \cite{Burgard2005}, which takes into account the costs of reaching a target and the utility of reaching that target. Whenever a target point is assigned to a specific robot team member, the utility of the unexplored area visible from this target position is reduced for the other team members. To emphasize, Burgard's and our utility functions have a different definitions, but lead to the same goal, minimum exploration time. The main difference between these two approaches is a target point assignment. This paper presents a new decentralized tagret point assignment process that enables all team members to be their own  \emph{leader} in contrast to the coordinated Burgard's approach, where the  \emph{leader} is a computer.

The rest of the paper is organised as follows. The exploration and mapping overview is presented in the next section. In Section III the exploration strategy is described. Simulation results are presented in Section IV, and in the final section a conclusion is given.

\section{EXPLORATION AND MAPPING OVERVIEW}
%\section{EXPLORATION STRATEGY DESCRIPTION}
%Change the title--Method overview
The laser scan and odometry sensor measurements of the mobile robot represent input data for a Simultaneous Localisation and Mapping (SLAM) module. In this work, for map building we use a submap-based graph SLAM method -- Google Cartographer \cite{Hess2016} -- which, however, in the simulation uses the ground truth mobile robot poses (\textit {perfect localisation}) from the Stage simulator \cite{stageweb} in order to eliminate the SLAM algorithm uncertainty. The module also implements frontier detection according to \cite{Orsulic2019}. This method has achieved impressive results in terms of wall-time per frontier update, what greatly speeds up our exploration and mapping process. 

Google Cartographer SLAM extended with frontier detection and filter module are centralized part of the exploration and mapping process (Fig. \ref{fig:exploration-strategy}). Centralized part generates filtered frontier points, inputs to $n$ equal decentralized strategies modules for $n$ mobile robots. During the exploration and mapping $n$ mobile robots communicate, exchange minimum information about frontier points in order to decide where to navigate and create common map of the unknown environment. Focus and scope of the paper is decentralization of the decision making process, using centralized SLAM algorithm and centralized filtered frontier point generator. Further, the algorithm for path planning and following navigates the mobile robot toward the target point. The process is over when the area is explored.  

\begin{figure}[t!]
    \centering\includegraphics[width=1.0\columnwidth]{./Pictures/diagram_exploration.pdf}
	\caption{Overall schematic diagram of the decentralized exploration and mapping process for $n$ mobile robots in the simulator. Google Cartographer and filter module (highlighted red) generate filtered frontier points that are centralized part of exploration and mapping process. Exploration strategy and path planning and navigation module (highlighted green) are decentralized parts that generate $n$ outputs and create common map.}
   \label{fig:exploration-strategy}
\end{figure}

Our multi-robot exploration and mapping process combine SLAM (Google Cartographer), a frontier detection algorithm, filter module and a decision making strategy. Each component can be easily changed or extended, without affecting an exploration and mapping process. For instance, the decentralized strategy can be replaced without affecting the frontier points detection. Similarly different types of frontier points detectors could be tested without affecting the behaviour of the decision making strategy. The same logic is valid for SLAM method and filter module. This makes our exploration and mapping process more flexible and modular. 

\subsection{SLAM} 
Most literature uses the ROS 'gmapping' package for generating the
map and localizing mobile robots \cite{Keidar2012}, \cite{Umari2017}. The 'gmapping' package implements a SLAM algorithm that uses a Rao-Blackwellized particle filter \cite{Grisetti2007}. 
On the other side, we use Google Cartographer SLAM algorithm \cite{Hess2016}, an open source available in \cite{cartographer}, which in our case generates ground truth map. Cartographer is an open-source 2D and 3D graph SLAM based on occupancy grid submaps. 
An extension to the Cartographer is a new fast frontier detection method developed in \cite{Orsulic2019}. 

\subsection{Frontier Detection}
A popular frontier-based exploration approach is introduced by Yamauchi \cite{Yamauchi1998}, where the robots explore an unknown environment and build a common map. Moreover, each robot heads to the center of mass of the closest frontier and frontier detection is performed only when the robot reaches its target. 
The frontier edges detection can be achieved using RRT (Rapidly Exploring Random Tree) algorithm by Umari and Mukhopadhyay \cite{Umari2017}. The RRT algorithm is biased towards unexplored regions and provides a general approach which can be extended to higher dimensional spaces. However, RRT algorithm proved not to be fast enough in instances when larger parts of the environments were explored, so we opted to use a dense frontier detection method \cite{Orsulic2019}. Orsulic has achieved great results in terms of wall-time per frontier update. Furthermore, this paper is the first usage of the dense frontier detection method in multi-robot exploration and mapping. 

\subsection{Filter Module}

The filter module receives the frontier points from frontier detector  and first clusters the points and stores only the centre of each cluster. The clustering process reduce the number of frontier points from frontier detection module, which are extremely close to each other. If such amount of points are sent to the decentralized exploration strategy module, there would be unnecessary consumption of computational resources, without additional information about the frontier. We use Hierarchical Agglomerative Clustering \cite{clutering} because it does not require predefined number of clusters. We only need to set distance threshold parameter, above which clusters will not be merged. Even though it has a time complexity of ${\displaystyle {\mathcal {O}}(n^{3})}$, it is suitable for our case because of the size of frontier points. Moreover, it is easy to implement. 

\subsection{Why Decentralized?}

Algorithms for assigning robots to target points can be grouped into centralized and decentralized algorithms. Centralized task assignment for a multi-robot system may be less practical due to communication limits \cite{Dias2000}, robustness issues \cite{Dias2006}, or time required for algorithm execution and scalability \cite{Julia2012}. In the centralized approach, each mobile robot receives tasks assigned from a single central \emph{leader} using a centralized planning algorithm. During communication between the central leader and the mobile robots, information about the mobile robot poses is shared in order to perform real time mobile robot task allocation. The central leader may be a computer or a robot. An advantage of centralized approach is that the optimal plans can be found \cite{Yan2011}. Nevertheless, this mechanism is ineffectual for large mobile robot teams.

In contrast to centralized approaches, in a decentralized approach, the mobile robots are completely autonomous in the exploration process. Each mobile robot has its own local knowledge of the world and can decide its future actions by taking into account its current context and task, its own capacities and the capacities of the other mobile robots, through a negotiation process \cite{Yan2013}. Moreover, it usually has better reliability, flexibility, adaptability and robustness \cite{Zlot2002}. There is no central planner in the decentralized multi-robot exploration problem, so mobile robots need to communicate and cooperate effectively to explore and map an unknown environment as soon as possible. 
In order to have more robust and flexible system we propose a strategy in which mobile robots are able to decide toward which target point to navigate, with the assumption that the mobile robots have knowledge of all target points, while we assume that only a single mobile robot can be assigned to a specific target point.  

In this paper we determine the target points for each mobile robot in the team using an objective function which is a combination of frontier point cost, utility of reaching the target point and a novel extension - the frontier occupancy function that makes mobile robot be dispersed throughout the environment. 

\section{DECENTRALIZED EXPLORATION STRATEGY} 

At the core of our paper is a decentralized strategy for multi-robot exploration and mapping. The mobile robots exchange information about frontier points under the assumption of a fully connected graph and event based communication. If we define a mission as a process from getting a target point to reaching the same point, then all mobile robots communicate with each other in the moments when the mission for a single mobile robot is over. It means that the rest of mobile robots should send the data even though their missions are not over and they are still navigating to the goal. Event based communication is triggered by mission accomplishments.   

The exploration task is performed by a team of $n$ mobile robots  \(\text{$\mathcal {R}$}\) = $ \{ 1, 2,..., N\}$ where the mobile robots do not have prior knowledge about the environment, i.e., the position of the boundaries and obstacles.  
Every mobile robot $i$ gets the list of frontier points  \(\text{$\mathcal {Y}$}\) = $ \{ 1, 2,..., M\}$ from filter module (Fig. \ref{fig:exploration-strategy}) in the moments when any of the mobile robots reaches the target point (when the mission is over). $M$ is the number of frontier points represented as a tuple ($x$, $y$), corresponding to the position in meter from the map origin. 

We define the frontier point weight $W$ as a function of cost $C$, utility $U$ and frontier occupancy $F$. Cost function $C$: $R_{xy}$ \(\rightarrow \text{$\mathbb{R}^{+}$}\) is a mapping from $R_{xy}$ to a positive real number.  If $i$ is the index of the mobile robot, and $j$ is the index of the frontier point, then $C_{ij}$ describes the cost of $i$th robot to visit the $j$th frontier point. The cost can be a function of time, energy or, like in our case, the estimated distance travelled by mobile robot to reach the target frontier point. The estimated distance is approximated using Euclidean distance between the mobile robot position $\boldsymbol{p_{i}}$ and the frontier point position $\boldsymbol{y_{j}}$:

\begin{equation}\small
    C_{ij}=d(\boldsymbol{p_{i}}, \boldsymbol{y_{j}}) = \sqrt{(p_{ix}-y_{jx})^{2}+(p_{iy}-y_{jy})^{2}}.
    \label{cost}
\end{equation}

The utility function $U_{ij}$: $R_{xy}$ \(\rightarrow \text{$\mathbb{R}^{+}$}\) returns a positive real number, and uses the occupancy grid \(\text{$\mathcal {M}$}\) to find the value for the specific frontier point. The cells of \(\text{$\mathcal {M}$}\) may be marked as explored space, unexplored space or obstacle. The utility function is proportional to the number of the unexplored cells $c$ within a fixed distance from the frontier point $y_{j}$ in the previous defined radius $r$: 

\begin{equation}
    U_{y_{j}} = \lambda_{u}c,
\end{equation}

where $\lambda_{u}$ is a constant determined experimentally. When the function $U_{y_{j}}$ is taken into account, the mobile robot will prefer frontier points that are surrounded by more unexplored space even if they are a little bit further. 
It is assumed that the mobile robot will detect all unexplored cells around the assigned frontier point after reaching it. 

The most important component is the frontier occupancy function $F_{ij}$, the 2-dimensional Gaussian function with the position of the mean in a frontier point and with the standard deviation \boldsymbol{\sigma} = \begin{bmatrix}
           r_{f} \\
           r_{f} 
   \end{bmatrix}.
   
 If the frontier point $y_{j}$, for which the mobile robot $i$ is calculating the weight, is in the range of radius $r_{f}$ from the position of an another mobile robot assigned point (\boldsymbol{y_{a}}); the value of the frontier occupancy function is calculated by Gaussian function, and zero otherwise:

\begin{equation}\label{eq:frontier_funct}
 F_{ij}=
\begin{cases} 
       \lambda_{f} e^{-\Big[\frac{(y_{jx} - y_{ax})^2}{2\sigma_{x}^2} + \frac{(y_{jy} - y_{ay})^2}{2\sigma_{y}^2}\Big]} & \text{if $d(\boldsymbol{y_{j}}, \boldsymbol{y_{a}})< r_{f}$}, \\
      \quad \quad \quad \quad \quad 0 & \text{otherwise}
   \end{cases}
\end{equation}

where $\lambda_{f}$ is an experimentally determined constant. 
The example of the frontier point $y_j$, assigned point $y_a$ and Gaussian function values inside the radius $r_f$ are shown in Fig. \ref{fig:gauss}.
The function $F_{ij}$ is used to prevent assigning the frontier point to the mobile robot B if that point is close to a point that is already assigned to mobile robot A.

\begin{figure}[t!]
	\centering
	\fbox{\includegraphics[width=0.95\columnwidth]{./Pictures/gauss.pdf}}
	\caption{A special scenario where the value of the frontier occupancy function $F$ is different from zero. Mobile robot is assigned to frontier point $y_a$, and follows the path to reach it. If another mobile robot calculates weight for the all current frontier points (green points), $F$ for frontier point $y_j$ is different from zero because $y_j$ is inside the radius $r_f$}.
	\label{fig:gauss}
\end{figure}

For each frontier point $y_{j}$, the weight $W_{ij}$ of the $i$-th mobile robot is calculated as: 
\begin{equation}
   {W}_{ij}= {C_{ij}} - {U_{y_{j}}} + {F_{ij}}.
   \label{weight}
\end{equation}

The weight matrix $\boldsymbol{W}$ ($N\times M$) is formed for $N$ mobile robots and $M$ frontier points: 

\begin{equation}
    \boldsymbol{W} = \begin{bmatrix}
    W_{00} & W_{01} & \hdots & W_{0j} & \hdots & W_{0M}\\
    W_{10} & \ddots & & & & \vdots\\
    \vdots & & \ddots & & &  \vdots \\
    W_{i0} & & & \ddots & & \vdots \\
    \vdots & & & & \ddots & \vdots\\
    W_{N0} & \hdots  & \hdots  & \hdots  & \hdots &    W_{NM}
    \end{bmatrix}.
\end{equation}

The mobile robots exchange information about their positions and current target points and make decisions for the future actions based on the exchanged information. The amount of exchanged data is thus reduced, which allows easier and faster communication.
The weight matrix $\boldsymbol{W}$ represents the input into the Hungarian algorithm that attempts to find an optimal assignment solution in polynomial time.

The Hungarian algorithm is described in \cite{Kuhn1955} and tested in \cite{Kulich2015}. Initially, the Hungarian algorithm assumes that the number of frontier points is the same as the number of mobile robots. Due to the fact that there are usually fewer mobile robots than frontier points, virtual mobile robots are added and then skipped during the process of assignment and exploration.

Let the matrix $X$ be the matrix of zeros and ones, where $X[i,j]=1$ iff the mobile robot $i$ is assigned to the frontier point $j$.
Than the optimal task assignment has weight:

\begin{equation}
     {\min_{X}}\ \Big(\sum_{i} \sum_{j} W_{ij}\ X_{ij}\Big),
\end{equation}

anticipating that minimisation of sum will ensure the dispersion of the mobile robots in the environment. 

\begin{algorithm}[t!]
\While{Unexplored}{
\If{Request}{
    Send position and current target point to the other mobile robots\;
    }
\If{Mobile robot $i$ has reached the previous target point}{
    Request positions and current target points from the other mobile robots\;
    \For{Each frontier point $y_{j}$}{
    Calculate the weight matrix $\boldsymbol{W}$\;
    Hungarian algorithm ($\boldsymbol{W}$)\;
    \textbf{return} Mobile robot $i$ is assigned to frontier point\;
    }
}
}
\label{algorithm1}
\caption{Decentralized strategy for mobile robot $i$}
\end{algorithm}


Our decentralized strategy executes during the robot motion and there are steps for each robot to be executed (\textbf{Algorithm 1}). All frontier points are visible to all mobile robots.
When mobile robot $i$ is assigned to frontier point according to line 10 in \textbf{Algorithm 1}, the mobile robot starts to follow the planned path and navigates to the target frontier point. Moreover, at the moment when the mobile robot $i$ reaches the target point (mission is over), a request is sent to the other mobile robots to get their positions and current target points, and to fill in the weight matrix $\boldsymbol{W}$, which is an input to Hungarian algorithm. The described process executes until the whole environment is explored, in other words, until a complete  map of the environment is generated. The Fig. \ref{fig:structure} illustrates the described steps and algorithm lines for a mobile robot team. 

\begin{figure}[h!]
    \centering\includegraphics[width=1.0\columnwidth]{./Pictures/struktura_vol1.pdf}
	\caption{An execution of the decentralized part of the Fig. \ref{fig:exploration-strategy}. Every mobile robot explores the environment following the steps from \textbf{Algorithm 1}. The blue robot accomplished the mission and requests weights from the rest team memebers (red ones). The strategy is the same for all mobile robots and starts in the moment when each of them reaches the target point.}
   \label{fig:structure}
\end{figure}


\section{SIMULATION RESULTS}

The proposed strategy is implemented and tested using the Robot Operating System (ROS) framework. Our decentralized strategy is compared with the coordinated strategy described in \cite{Burgard2005}. %Describe Burgard here

\begin{figure}[b]
    \centering
	\begin{minipage}{1.0\columnwidth}
	   \centering
	   \includegraphics[width=\columnwidth]{./Pictures/belgioioso.pdf}
	   \caption*{a)} 
	\end{minipage}
\hfill 	
	\begin{minipage}{1.0\columnwidth}
	   \centering
	   \includegraphics[width=\columnwidth]{./Pictures/map.pdf}
	   \caption*{b)} 
	\end{minipage}
\caption{Maps used for simulation experiments. a) Original map: Map from \cite{fr_dataset}. b) Google Cartographer map: The map generated using Google Cartographer SLAM algorithm with previous hand corrections at incomplete parts of the original map.}
\label{scenario}
\end{figure}
We implemented coordinated Burgard's algorithm so that the cost of reaching the frontier point is also proportional to the Euclidean distance between the robot and that point and the utility of a frontier point depends on the number of robots that are moving to that point. Each point is eventually a cell of an occupancy grid map (represents the environment) that contains a numerical value and describes if the corresponding area in the environment is covered by an obstacle or not. The cell are then classified as occupied or free. The frontier detector and filter module are the same for both algorithm implementations (coordinated Burgard's and our decentralized). 


\subsection{Simulation Setup}
Simulations were carried out using the Stage simulator \cite{stageweb}, which provides realistic robot movements inside the loaded environment image. The ROS Navigation stack is used to control and direct the mobile robots towards exploration goals. 
Scenario used in the simulation is Belgioioso Castle, available in \cite{fr_dataset} and shown in Figure \ref{scenario}. It is a challenging and a typical office like scenario with a free space area of approximately 225 ${m}^2$. For the simulation, we use a model of Pioneer P3-DX with maximum speed of 1.3 m/s, laser range 20 m and 360$^{\circ}$ laser scan window. 
We tested the algorithms with teams of two, three and five mobile robots. The number of mobile robots is in our case limited because of the complex simulation setup and computational requirements. Thus we use the appropriate size of the simulation scenario. The robots start off in random positions within this world. The results are presented as average of 10 runs for each set. Constants $\lambda_{u}$, $\lambda_{f}$, $r$ and $r_{f}$ are experimentally determined and set to  0.9, 1.2, 0.5, 3.0 respectively.

\begin{figure}[t]
    \centering
	\begin{minipage}{1.0\columnwidth}
	   \centering
	   \includegraphics[width=\columnwidth]{./Pictures/burgard_results.pdf}
	   \caption*{a)} 
	\end{minipage}
\hfill 	
	\begin{minipage}{1.0\columnwidth}
	   \centering
	   \includegraphics[width=\columnwidth]{./Pictures/decent_results.pdf}
	   \caption*{b)} 
	\end{minipage}
\caption{Simulation results for a) coordinated Burgard's and b) decentralized strategy tested on two, three and five mobile robots.}
\label{fig:results}
\end{figure}


\subsection{Simulation Results}

During the exploration and mapping process covered area in time is recorded. The comparison of the coordinated Burgard's strategy and our decentralized strategy is shown in Fig. \ref{fig:results} using an indicator called \textit{Coverage Ratio (CR)} which addresses percentage of the accessible terrain covered by the mobile robot team. It is calculated as:  \( \frac{\text{explored cells} \cdot 100}{\text{accessible cells}} \).

First of all, it is interesting to see how the coverage evolves over time. Fig. \ref{fig:results} shows \textbf{CR} over time for both coordinated Burgard's and decentralized multi-robot exploration strategies in the described scenario. We report the time it takes to cover 50, 75, 90 and 98 percent of the environment. In the both scenarios, our decentralized strategy needed less time to explore the same percentage of the environment. Furthermore, increase in performance was greater for larger number of mobile robots. Using our decentralized strategy, it took \textbf{7.8 \% , 15.3 \% and 32.6 \%} less time for two, three and five mobile robots respectively to explore the environment compared to coordinated Burgard's strategy. This results show that our decentralized strategy performs significantly better than coordinated Burgrad's, especially for five mobile robots.

When we compare the average exploration time for the both strategies it shows that three mobile robots performs better than two, as well as having five mobile robots instead of three. An area size plays an important role in the optimal number of mobile robots. Due to the relative small scenario and laser range, we assume that more mobile robots will make the area overcrowded. Video recordings for simulation with five mobile robots can be found on \cite{playlist}.


%%%%%%%%%%%%%%%%%%%%%%%%%%%%%%%%%%%%%%%%%%%%%%%%%%%%%%%%%%%%%%%%%%%%%%%%%%%%%%%%

\section{CONCLUSION AND FUTURE WORK}
We have presented a modular approach to autonomous decentralized multi-robot exploration and  mapping which is, besides, not restricted to Google Cartographer SLAM and dense frontier detection. This strategy has demonstrated improved behaviour compared to the coordinated Burgard's strategy in term of exploration time. Another contribution of this work is a repository (available on Github\footnote{\href{https://github.com/larics/decentralized_multi_robot_exploration}{https://github.com/larics/decentralized\_multi\_robot\_exploration}}) that includes developed and implemented decentralized multi-robot exploration strategy as well as our implementation of the Burgard's algorithm. 

Despite these encouraging results, there are several aspects which could be improved. One of the interesting research direction is to consider decentralized map creating and complex data sharing. Also, an improvement can be a simpler simulator with more mobile robots, which we believe will approve better behaviour of our decentralized strategy over coordinated Burgard's strategy. Another direction can be an extension of the algorithm to cope with a limited communication range of the mobile robots, which will probably improve overall performance. Additionally, we want to investigate scenarios in which the robots may malfunction or break or in which the environment changes over time.

%%%%%%%%%%%%%%%%%%%%%%%%%%%%%%%%%%%%%%%%%%%%%%%%%%%%%%%%%%%%%%%%%%%%%%%%%%%%%%%%
\addtolength{\textheight}{-12cm}   % This command serves to balance the column lengths
                                  % on the last page of the document manually. It shortens
                                  % the textheight of the last page by a suitable amount.
                                  % This command does not take effect until the next page
                                  % so it should come on the page before the last. Make
                                  % sure that you do not shorten the textheight too much.
\nocite{*}
\bibliographystyle{ieeetr}
\bibliography{Bibliography/decentralized}

\end{document}
